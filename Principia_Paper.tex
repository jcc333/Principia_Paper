\documentclass{article}
\usepackage[utf8]{inputenc}
\usepackage{comment}
\begin{document}
\title{The Necessity for Functions to Accept Arguments of a Certain Type \textit{(paper on Section IV)}, or, "this sentence is a statement created by using a proposition as the argument to a function"}
\author{James Clemer}
\date{Februrary, 2013}

\maketitle 

\paragraph{}
In his \textit{Principia Mathematica}, Bertrand Russell defines the theory of types ($ToT$) in order to reinforce one of the Fregian account's important weaknesses: the Russell Paradox.\footnote[1]{Concisely, let $R=\{x|x\not\in x\}$, then $R \not\in R \longleftrightarrow R \in R$} While specifying the theory of types, in section iv of the introduction to part 2, Russell argues that any given function can only accept arguments of a certain type. It seems to Russell that if this were not included in $ToT$, the theory could not adequately address the paradoxes it ought to solve. Ultimately, restricting the types of function arguments, as Russell does in the in his \textit{Principia}, solves problems for logicism and helps ameliorate the paradoxes which plagued Frege's account. That said, Russell may be too restrictive in defending this corner of $ToT$, especially for his final example: the sensibility of functions over propositions.
\paragraph{}
According to Russell, a function has to accept arguments of a specific type.
We can call this the ‘type-strictness’, or ‘strict typing’ of functions.
Russell’s hierarchy of types grows with the ambiguity of an entity. So a 0th order entity can be defined as a statement with no variables, no ambiguity. A 1st order function becomes definite when given definite entities for its variables (the function “$x$ is human”, for example), and an $n$th-order entity is built from lower order elements likewise\footnote[2]{Bertrand Russell, "Logic and Knowledge",pg.75}.
For Russell to demonstrate the need for type-strictness, he ought to show that a function application to a too low typed argument or too high typed-argument is absurd.
\paragraph{}
To begin advocating type-safety, Russell asks us first to accept that $\phi(\phi\hat x)$ is absurd, based upon arguments from sections i-iii \footnote[3]{"The considerations so far adduced in favor of the view that cannot significantly as argument anything defined in terms of the function itself..."-\textit{Pricipia},pg.47}. In those sections, he presents the vicious cycle principle \footnote[4]{"Whatever involves \textit{all} of a collection must not be one of the collection.", \textit{Principia Mathematica}, pg.} and argues for the absurdity of $\phi(\phi\hat x)$ based on the principle. But it seems that Russell does not need to use his arguably ad hoc vicious cycle principle. Instead, he could demonstrate the necessity of the principle, at least in this limited circumstance. Russell can assert that each function’s argument has a type, even without proving the importance of types to skeptics, just by constructing types. 
\paragraph{}
Russell can define some procedure for generating a logical structure’s type, even without the agreement that types matter. Russell lets the elementary units of the universe form the lowest level of his hierarchy, and the other levels are built by combination of the levels below.Then, a function like $\phi(\phi\hat x)$ gives us serious problems. Types are recursively defined above, and anything but nonsense will either have a type. Assume, contrary to Russell, $\phi\hat x$ can meaningfully accept arguments of any type.
Then $\phi\hat x$’s arguments each have some type, but what about $\phi(\phi\hat x)$? Its type, since it can accept any argument, is ambiguous. Yet, if $\phi(x)$ has some variable type $\alpha$ dependent on its argument, then what would $\phi(\phi\hat x)$’s type be? It would have to remain ambiguous, so $\phi(\phi\hat x)$ cannot be definite, and cannot appear as a value of $\phi\hat x$, and $\phi(\phi\hat x)$ is nonsense.
\paragraph{}
From the assertion that “$\phi(\phi\hat x)$ is absurd”, Russell generalizes the claim to “for any $\phi$ and $\psi$ with some $a$ such that $a$ is a value of $\phi\hat x$ and of $\psi\hat x$, the expression $\phi(\psi\hat x)$ is absurd”\footnote[5]{"not only is it impossible for a function $\phi\hat z$ to have itself or anything derived from it as an argumenit; but that, if $\psi\hat z$ is another function such that there are arguments $a$ for which both "$\phi a$" and "$\psi a$", then $\psi\hat z$ and anything derived from it cannot significantly be argument to $\phi\hat z$"-\textit{Principia}, pg.47}.
Russell's functions are ambiguities unresolved\footnote[6]{"...a function is essentially an ambiguity..."-\textit{Principia}, pg.47}. That is, a function without an argument is ambiguous, indefinite. A function given an appropriate argument means something definite.
But Russell is cagey here; he doesn’t explicitly write why the ambiguity of functions means that $\phi(\psi\hat x)$ is absurd when there is a shared argument $a$. Reading into his intentions, it seems like he means that for any $\phi\hat x$ and $\psi\hat x$ which accept arguments of the same type, $\phi(\psi\hat x)$ will yield an ambiguous value by the same reasoning that $\phi(\phi\hat x)$ is ambiguous.
Russell intends his two examples to demonstrate that an expression breaking strict-typing cannot create anything significant. Essentially, he gives us the following two examples: first, “$(x).\phi x$ cannot sensibly accept a non-function”, and then, “‘$\phi\hat x$ is a man’ is nonsense.”\footnote[8]{"Thus, '$(x).\phi x$', which we have already considered, is a function of $\phi\hat x$..."\textit{Principia}, pg.47}\footnote[9]{"Take, \textit{e.g.}, '$x$ is a man' consider '$\phi x$ is a man'. ..."\textit{Principia}, pg.48}.
The first example should show us that an argument from too low a type gives us something senseless.
It seems reasonable, but there’s a potential worry for Russell. He translates $(x).\phi x$ as “$\phi x$ in all cases [of $x$]” and then argues that a non-function in $\phi x$’s place is senseless, because there are no cases of $x$ in that $p$. But with some proposition $p$, $(x).p$ could mean “$p$ in every case of $x$". In which case, $p$ can be anything which must be true, regardless of other facts. One example of such a $p$ would be a statement like $p\leftrightarrow p$, which is true regardless of other facts.
Russell could argue that $(x).p\leftrightarrow p$ still means nothing, but it seems like $(x).p$ is just a statement of $p$'s tautological truth. That sort of statement may be reducible to just $p$, but it ought to be allowable in a logical system, regardless.
The second example works to the effect that “a first-order function cannot accept another function meaningfully”. 
Russell argues that a function does not denote a definite object, so applying a function over too high-typed an argument will not indicate anything worthwhile: the ambiguity of the argument will not resolve itself to something definite when it's given to the function.
So “$\phi\hat x$ is a man” cannot be definite in the way that “$\phi(some\ x)$ is a man” can. To the contrary, a function must be applied until its ambiguity is resolved to serve as a definite object, at least in Russell’s formulation of functions and the definite.
\paragraph{}
Finally, Russell offers the following caveat: that a proposition, despite being definite, cannot generally serve as argument to a function\footnote[10]{"We need here a new objection [to, '$(x).\phi x$ is a man'], namely the following: A proposition is not a single entity, but a relation of several;..."-\textit{Principia}, pg.48}.
An example of what Russell means follows. Let $s$ be “‘Socrates is a man’ is a proposition”. $s$ is a statement created by placing a proposition as the argument to a function. It appears that $s$ asserts something definite: the meaningfulness of “Socrates is a man”. So it may serve as a counterexample to Russell’s notion that a proposition cannot be the argument to a function; unfortunately, Russell's point at the close of section iv is unclear.
\paragraph{}
Russell argues that the proposition in $s$ is itself a relation, and so $s$ will only be significant if reduced to a statement about the terms of that proposition. He then argues that because “$p$ is a man”, where $p$ is a proposition cannot be so reduced, that “$\{(x).\phi x\}$ is a man” is meaningless. 
A worry here, though, is that just because “$\{(x).\phi x\}$ is a man” is meaningless, it is not necessarily true that any sentence of the form “$\{(x).\phi x\} ...$” is meaningless, which Russell seems to be implying.
\paragraph{}
On the other hand, it may just be that Russell argues that any meaningful sentence with a form like “$\{(x).\phi x\}$ is a man” can be reduced to another sentence relating to $\{(x).\phi x\}$’s variables. $s$ certainly follows that rule, in that “$x$ is a proposition” reasons about the contents of $x$.
\paragraph{}
Russell’s requirement of strictly typed functions in set theory does escape the hairy paradoxes of naive set theory, but it does so at the possible risk of being unnecessarily restrictive. If Russell means that anything of the form “$\{(x).\phi x\}$ is a man” has no meaning, he is unnecessarily restricting our ability to express facts about propositions. On the other hand, if he meant that for such propositions to be meaningful, they must address their arguments, there is no clear problem.

\end{document}
%%%%%%%%%%%%%%%%%%%%%%%%%%%%%%%%%%%%%%%%%%%%%%%%%%%%%%%%%
%                  scratchwork                          %
%%%%%%%%%%%%%%%%%%%%%%%%%%%%%%%%%%%%%%%%%%%%%%%%%%%%%%%%%
\begin{comment}
If we accept Russell's earlier points, then we explicitly accept his vicious circle principal, and the invariant that a function $\phi \hat z$ cannot contain $\phi \hat z$, nor can it operate over itsef($\phi (\phi \hat z)$). Further, Russell argues that if 2 functions $\phi$ and $\psi$ have some common values $a$ in their domains, then $\phi (\psi \hat z)$ is also senseless. Russell assumes that for a function to be sensible, it must be definite; with no remaining ambiguity. Russell argues that this claim proves that $\phi (\psi \hat z)$ is insignificant.
\paragraph{}

In "section IV", Russell defends this claim based on a series of earlier points, especially his vicious circle principal\footnote[2]{"Whatever involves \textit{all} of a collection must not be one of the collection."}, and the invariant that a function $\phi \hat z$ cannot contain $\phi \hat z$, nor can it operate over itsef($\phi (\phi \hat z)$)\footnote[3]{Recall that Russell lets $\phi(x)$ be some proposition generated by the unapplied function $\phi\hat x$, and $(x).\phi x$ is the set of all $x$s such that $\phi(x)$.}. After reiterating the later idea, Russell then proposes generalizing that claim, writing,
   \begin{quote}
   "... not only is it impossible for $\phi\hat x$ to have itself or anything derived from it as argument, but that, if $\psi\hat x$ is another function such that there are arguments $a$ for which both "$\phi a$" and "$\psi a$" are significant, then $\psi\hat z$ and anything derived from it cannot significantly be argument to $\phi\hat z$." -(\textit{Principia Mathematica}, pg.47)
   \end{quote}
   This generalization restricts set theory, ensuring that a function cannot apply over a function of the same level in the hierarchy. Russell offers the following justification for his restriction: that a function is an ambiguity of some sort, and for it to be definite, that ambiguity must be resolved. Russell offers a few examples to clarify his point. First, Russell indicates that $(x).(\phi\hat x)$ is a function with $\phi\hat x$ as its variable, and if $\phi\hat x$ is not a function, $(x).(\phi\hat x)$ looses all meaning. If $\phi\hat x$ is not a function, but some proposition $p$, then $(x).(\phi\hat x)$ becomes $(x).p$, or, "Each $x$ such that $p$". This is nonsensical to Russell because a definite proposition does not involve variables, and so the involvement of $x$ seems insignificant. To the contrary, the function $(x).p$ might be somewhat uninteresting, but it is not insignificant: If $p$ is true then for all $x$, $p$, so it seems that $(x).p$ is all items in the universe. If, however $p$ is false, then for all $x$, $\not p$, so $(x).p$ is none of the items in the universe. Russell might find issue with $p$'s ignoring $x$, but that ignorance is just a particularly lax restraint upon $x$, not a reason to exclude $(x).p$ from set theory. In his defense however, Russell might argue that $x$, without being specified as a variable in a function, denotes nothing.
   If the previous example constitutes an argument from too low in the hierarchy of types, the next provides us with the opposite. Russell claims that where a non-function argument yields a definite proposition, a function-as-argument yields something senseless, with the example of giving the argument $\phi \hat x$ to "$x$ is a man". Russell notes that $\phi\hat x$ is, as a function, ambiguous, and that in a function which accepts a non-function as its argument, the ambiguity of $\phi\hat x$ remains unresolved. So "$\phi\hat x$ is a man" cannot mean anything, because one cannot determine that $\phi\hat x$ is a man\footnote[4]{"A function, in fact, is not a definite object, which could be or not be a man; it is a  mere ambiguity awaiting determination, and in order that it may occur significantly it must receive the necessary determination, which it obviously does not receive if it is merely substituted for something appropriate in a proposition".-(\textit{Principia Mathematica},pg.48)}.
   \end{comment}

